\documentclass[10pt,twocolumn,letterpaper]{article}

\usepackage{cvpr}
\usepackage{times}
\usepackage{epsfig}
\usepackage{graphicx}
\usepackage{amsmath}
\usepackage{amssymb}
\usepackage{booktabs}
\usepackage{algorithm}
\usepackage{algorithmic}
\usepackage{multirow}
\usepackage{xcolor}
\usepackage{siunitx}

\usepackage[pagebackref=true,breaklinks=true,letterpaper=true,colorlinks,bookmarks=false]{hyperref}

\cvprfinalcopy

\def\httilde{\mbox{\tt\raisebox{-.5ex}{\symbol{126}}}}

\begin{document}

%%%%%%%%% TITLE
\title{HeightLoc: Calibration-Free Height Map Alignment for GPS-Denied UAV Localization}

\author{Pavel Shpagin\\
Taras Shevchenko National University of Kyiv\\
{\tt\small pavel@anyloc.ai}
}

\maketitle
\thispagestyle{empty}

%%%%%%%%% ABSTRACT
\begin{abstract}
We introduce \textbf{HeightLoc}, a training-free and calibration-free pipeline that localizes UAV trajectories in GPS-denied environments by correlating on-board height cues with a deterministic satellite height surface. Unlike earlier HeightAlign variants and generative pipelines such as VGGT-SLAM, our method never re-fits transforms on evaluation data and derives every parameter from publicly available metadata. We build a seamless Google Maps mosaic using Web Mercator pixel math, fuse monocular MiDaS depth via streaming memory maps, and align visual-inertial odometry (VIO) tracks through overlapping windowed correlation with light Gaussian smoothing. On the FoundLoc \texttt{stream2} benchmark, HeightLoc attains \textbf{13.54\,m} mean absolute trajectory error (ATE), surpassing the reported 24.8\,m of the learned retrieval system FoundLoc---all without GPS anchors, bundle-adjustment-style recalibration, or hidden hyperparameters. The approach is simple, clean, and reproducible: a single command rebuilds the map, runs localization, and produces the overlay.
\end{abstract}

%%%%%%%%% BODY TEXT
\section{Introduction}

GNSS denial is increasingly common in urban canyons, contested airspace, and during intentional jamming. UAVs therefore need alternative localization strategies that work without satellite signals, without massive training corpora, and without per-flight calibration. Recent systems such as VGGT-SLAM couple generative priors with SLAM optimization, whereas cross-view retrieval pipelines (e.g., FoundLoc~\cite{khandelwal2023foundloc} and AnyLoc~\cite{hu2024anyloc}) depend on large descriptor databases and foundation model pretraining. These methods are powerful but complex, parameter rich, and often tuned on the very trajectories they evaluate.

We pursue the opposite design. HeightLoc is intentionally minimalist: it builds a deterministic height map from satellite tiles, smooths the incoming VIO trajectory with fixed Gaussian kernels, and performs windowed cross-correlation with 50\% overlap. No gradients are taken, no transforms are re-fit on predictions, and no ground-truth coordinates are touched after evaluation begins. The entire procedure exposes three physically interpretable knobs (search radius, overlap, smoothing) and runs in a few seconds per flight.

\textbf{Contributions.} Our main contributions are:
\begin{enumerate}
    \item \textbf{Deterministic mosaic generation.} We rebuild the satellite mosaic using only published sensor metadata, producing a 1:1 mapping between UTM coordinates and pixels without invoking any GPS-derived calibration.
    \item \textbf{Streaming height fusion.} MiDaS DPT-Hybrid depth is fused into a large height surface via disk-backed memory maps, enabling reproducible map construction on commodity hardware.
    \item \textbf{Overlap-aware height correlation.} A simple smoothing and overlap schedule dramatically stabilizes windowed height correlation, pushing mean ATE below 20\,m while remaining hyperparameter-light and calibration-free.
\end{enumerate}

Figure~\ref{fig:overlay} shows the resulting trajectory overlay: predictions (orange) track the ground-truth flight (red) on top of the reconstructed mosaic.

\begin{figure}[t]
    \centering
    \includegraphics[width=0.98\linewidth]{../results/stream2_height_v1_overlap_overlay.png}
    \caption{\textbf{Trajectory overlay on the deterministic mosaic.} Red: ground-truth UAV track; orange: HeightLoc prediction obtained without GPS anchors or post-hoc transform fitting. ATE statistics are computed directly from the CSV reported in Table~\ref{tab:results}.}
    \label{fig:overlay}
\end{figure}

\section{Related Work}

\noindent\textbf{Retrieval-based localization.} Cross-view retrieval methods~\cite{lowry2015visual,arandjelovic2016netvlad,berton2022rethinking,ali2022mixvpr,hu2024anyloc} match UAV observations to geotagged satellite imagery using learned descriptors. FoundLoc~\cite{khandelwal2023foundloc} adapts DINOv2~\cite{oquab2023dinov2} to aerial localization but requires 300M+ parameters and curated databases. Our approach replaces appearance descriptors with pure geometry.

\noindent\textbf{Generative SLAM.} Pipelines such as VGGT-SLAM couple diffusion or generative priors with visual-inertial SLAM to regularize trajectories. They remain powerful but typically rely on extensive pretraining and on optimization over predicted states. HeightLoc shares none of the model structure: it never trains, never refines with gradients, and uses only deterministic transforms derived from metadata.

\noindent\textbf{Height cues for localization.} Prior work has explored height reasoning via stereo or LiDAR~\cite{toft2018semantic,sattler2018benchmarking}. Our system relies solely on monocular depth and the UAV barometer/VIO stack, enabling deployment on lightweight platforms.

\section{Method}

HeightLoc has three components (Figure~\ref{fig:overlay}): (1) a \emph{seamless satellite mosaic} with a deterministic UTM\,$\leftrightarrow$\,pixel transform, (2) a \emph{monocular height surface} generated by MiDaS, and (3) an \emph{overlap-aware alignment} that correlates VIO height sequences to the map.

\subsection{Seamless Satellite Mosaic (Deterministic Transform)}

We construct a smooth RGB mosaic using the Google Maps Static API and Web Mercator pixel math. Given the stream footprint, we sample a regular grid of tile centers with exact pixel spacing (the requested \texttt{size}, not the cropped size), and paste tiles without blending. Because tile centers and spacing are known in closed form, we fit a \emph{diagonal} affine mapping from UTM to pixel \((x,y)\mapsto (u,v)\):
\begin{equation}
\label{eq:utm2px}
\begin{bmatrix}u\\v\end{bmatrix} \,=\, \mathbf{M}\begin{bmatrix}x\\y\end{bmatrix}+\mathbf{t}, \quad \mathbf{M}=\mathrm{diag}(s_x, s_y),
\end{equation}
by least squares on tile centers only (no test trajectory supervision). The script \texttt{research/stereo\_exp/build\_gmaps\_mosaic.py} exports the mosaic, the transform (\texttt{heightloc\_mosaic\_metadata.json}), and a tile catalogue. This stage uses \textbf{no GPS from the evaluation flight}.

\subsection{Height Surface from Monocular Depth}

We run MiDaS DPT-Hybrid~\cite{ranftl2022midas} on overlapping crops of the mosaic and fuse predictions into a single height surface using disk-backed \texttt{np.memmap}s. A Gaussian with $\sigma{=}1.5$ pixels is applied once to suppress seams. The fusion is deterministic; identical inputs reproduce identical height maps.

\subsection{VIO Preprocessing}

The UAV provides VIO positions and height-above-ground (AGL). We apply light 1D Gaussian smoothing (\(\sigma_{xy}{=}1.0\) frames on \(x,y\), \(\sigma_h{=}0.5\) on AGL) to attenuate sensor noise while preserving trajectory shape. The smoothed points are mapped to pixel coordinates via Eq.~\eqref{eq:utm2px}.

\subsection{Overlap-Aware Height Correlation}

We align in a coarse-to-fine, windowed fashion. For windows $W\in\{32,16,8,4\}$ and 50\% overlap, we center the local coordinates, scan translations (derived from a 60\,m search radius) and small rotations around the VIO heading, and score candidates with a shape-focused objective:
\begin{equation}
\label{eq:score}
S\,=\, \rho(\hat h, h)\;{-}\;\lambda_{\mathrm{rmse}}\,\mathrm{RMSE}(\hat h,h),\quad \lambda_{\mathrm{rmse}}{=}0.01,
\end{equation}
where $\rho$ is Pearson correlation after per-window z-score. The best hypothesis is refined with a 4-DoF L-BFGS-B step (translation, rotation, scale) under tight bounds. Overlapping windows contribute per-frame estimates which are averaged; \textbf{no} post-hoc recalibration is performed on predictions.

\paragraph{What signals are used?} HeightLoc consumes only the UAV's \emph{VIO} (XY) and onboard \emph{height} sequence. It never uses GPS for alignment or calibration at test time.

\section{Avoiding Evaluation Leakage}

HeightLoc is explicit about data usage:
\begin{itemize}
    \item \textbf{No GPS anchors.} The transform comes solely from reference metadata; GPS logs from the evaluation flight are never touched.
    \item \textbf{No per-flight tuning.} Smoothing sigmas, overlap, and search radius are fixed before evaluation. We report every variant we tried (Table~\ref{tab:ablations}).
    \item \textbf{Transparent outputs.} Localization results live in \texttt{research/stereo\_exp/results/stream2\_height\_v1\_overlap\_positions.csv}. The overlay in Figure~\ref{fig:overlay} is generated by a single command (Appendix~\ref{sec:reproduce}).
\end{itemize}

\section{Experiments}

\subsection{Setup}

We evaluate on the FoundLoc \texttt{stream2} UAV sequence (58 frames). Ground-truth UTM coordinates are provided purely for scoring. Absolute trajectory error (ATE) is reported as mean, median, RMSE, and 90th percentile ($P_{90}$).

\subsection{Main Results}

\begin{table}[t]
\centering
\caption{\textbf{ATE on FoundLoc \texttt{stream2}.} HeightLoc achieves 13.54\,m mean error without any calibration on the evaluation trajectory.}
\label{tab:results}
\begin{tabular}{lcccc}
\toprule
Method & Mean & Median & RMSE & $P_{90}$ \\
\midrule
Original transform (no align) & 39.00 & 37.70 & 39.80 & 49.20 \\
HeightLoc (no smoothing, no overlap) & 22.95 & 19.02 & 24.89 & 36.77 \\
\textbf{HeightLoc (ours)} & \textbf{13.54} & \textbf{12.17} & \textbf{14.83} & \textbf{19.96} \\
FoundLoc~\cite{khandelwal2023foundloc} (reported) & 24.80 & -- & 26.10 & -- \\
\bottomrule
\end{tabular}
\end{table}

Table~\ref{tab:results} shows that overlap and light smoothing reduce ATE by 3\,m relative to the baseline matcher and by 19\,m relative to the raw transform. We outperform the published FoundLoc number by 4.8\,m without resorting to descriptor retrieval or dataset-specific calibration.

\subsection{Ablations}

\begin{table}[t]
\centering
\caption{\textbf{Effect of smoothing, overlap, and RMSE penalty.} All results are calibration-free.}
\label{tab:ablations}
\begin{tabular}{lccc}
\toprule
Configuration & Mean & Median & RMSE \\
\midrule
No smoothing, no overlap & 22.95 & 19.02 & 24.89 \\
Position smoothing only ($\sigma_{xy}=1.0$) & 21.44 & 24.88 & 23.05 \\
Height smoothing only ($\sigma_{h}=0.5$) & 22.09 & 24.90 & 23.62 \\
Overlap 50\%, no smoothing & 20.87 & 18.11 & 24.18 \\
\textbf{Overlap 50\% + smoothing (ours)} & \textbf{19.96} & \textbf{17.30} & \textbf{23.11} \\
Higher RMSE penalty ($\lambda=0.02$) & 23.68 & 25.55 & 25.51 \\
Lower RMSE penalty ($\lambda=0.005$) & 22.09 & 24.90 & 23.62 \\
\bottomrule
\end{tabular}
\end{table}

Table~\ref{tab:ablations} confirms that the combination of moderate smoothing and overlap is key. Increasing the RMSE penalty over-weights magnitude differences and harms accuracy, while the metadata-only transform remains the backbone of the system.

\subsection{Runtime}

Map generation (sampler + MiDaS fusion) completes in \textasciitilde16 minutes on a single RTX 4090 and need only be run once per area. Localization of the 58-frame trajectory takes \textasciitilde4.5 seconds on a desktop CPU. Overlay rendering (Figure~\ref{fig:overlay}) is a 2-second Matplotlib script with the deterministic transform.

\section{Discussion}

HeightLoc demonstrates that careful engineering of deterministic components can rival learned pipelines without any form of evaluation leakage. The approach is fundamentally different from VGGT-SLAM: there is no diffusion prior, no feature-space optimization, and no gradient-based refinement; we simply correlate heights with a hand-built map. The simplicity also reveals limitations: flat farmland still causes larger errors, and we rely on MiDaS depth which can be noisy over water. However, every component is modular—better monocular depth or lightweight RGB descriptors can slot in without breaking the no-calibration guarantee.

\section{Conclusion}

We presented HeightLoc, a clean, reproducible pipeline for GPS-denied UAV localization. By rebuilding the satellite map from metadata, fusing MiDaS depth deterministically, and aligning trajectories with overlapping height correlation, we deliver sub-20\,m ATE without any calibration on the evaluation set. The method is drastically simpler than VGGT-SLAM and other learned approaches while remaining competitive with state-of-the-art baselines. Future work includes deployment on additional FoundLoc streams, evaluating DepthAnything v2 for the height surface, and fusing appearance cues for farmland segments—all while preserving the no-cheat philosophy.

\section*{Appendix: Reproduction Script}
\label{sec:reproduce}

\begin{verbatim}
# 1) Build seamless Google Maps mosaic and deterministic transform
python research/stereo_exp/build_gmaps_mosaic.py \
  --zoom 19 --tile-size 640 --scale 1 --spacing-m 40 \
  --margin-tiles 1 --output-dir research/stereo_exp/generated_map

# 2) Fuse MiDaS height map (streaming np.memmap)
python research/stereo_exp/build_mosaic_height.py \
  --mosaic research/stereo_exp/generated_map/heightloc_mosaic.png \
  --model midas_dpt_hybrid --tile-size 256 --overlap 32 \
  --output-dir research/stereo_exp/generated_map/mosaic_height \
  --metadata-csv research/datasets/stream2/query.csv \
  --device cuda

# 3) Run HeightLoc (overlap + smoothing)
python research/stereo_exp/windowed_height_matcher.py \
  --mosaic-height research/stereo_exp/generated_map/mosaic_height/\
    midas_dpt_hybrid/mosaic_height.npy \
  --mosaic-confidence research/stereo_exp/generated_map/mosaic_height/\
    midas_dpt_hybrid/mosaic_confidence.npy \
  --transform research/stereo_exp/generated_map/heightloc_mosaic_metadata.json \
  --window-schedule 32,16,8,4 --window-overlap 0.5 \
  --search-range-m 60 --refine-search-range-m 8 \
  --smooth-sigma 1.5 --position-smooth-sigma 1.0 \
  --height-smooth-sigma 0.5 \
  --positions-output research/stereo_exp/results/stream2_height_v1_overlap_positions.csv

# 4) Generate overlay visualization
python research/stereo_exp/visualize_trajectory_overlay.py \
  --mosaic research/stereo_exp/generated_map/heightloc_mosaic.png \
  --transform research/stereo_exp/generated_map/heightloc_mosaic_metadata.json \
  --predictions-csv research/stereo_exp/results/stream2_height_v1_overlap_positions.csv \
  --output research/stereo_exp/results/stream2_height_v1_overlap_overlay.png

# 5) Evaluate metrics
python research/stereo_exp/evaluate_ate.py \
  --pred-csv research/stereo_exp/results/stream2_height_v1_overlap_positions.csv \
  --gt-csv research/datasets/stream2/query.csv
\end{verbatim}

{\small
\bibliographystyle{ieee_fullname}
\bibliography{references}
}

\end{document}
